%% Generated by Sphinx.
\def\sphinxdocclass{report}
\documentclass[letterpaper,10pt,english]{sphinxmanual}
\ifdefined\pdfpxdimen
   \let\sphinxpxdimen\pdfpxdimen\else\newdimen\sphinxpxdimen
\fi \sphinxpxdimen=.75bp\relax
\ifdefined\pdfimageresolution
    \pdfimageresolution= \numexpr \dimexpr1in\relax/\sphinxpxdimen\relax
\fi
%% let collapsible pdf bookmarks panel have high depth per default
\PassOptionsToPackage{bookmarksdepth=5}{hyperref}

\PassOptionsToPackage{booktabs}{sphinx}
\PassOptionsToPackage{colorrows}{sphinx}

\PassOptionsToPackage{warn}{textcomp}
\usepackage[utf8]{inputenc}
\ifdefined\DeclareUnicodeCharacter
% support both utf8 and utf8x syntaxes
  \ifdefined\DeclareUnicodeCharacterAsOptional
    \def\sphinxDUC#1{\DeclareUnicodeCharacter{"#1}}
  \else
    \let\sphinxDUC\DeclareUnicodeCharacter
  \fi
  \sphinxDUC{00A0}{\nobreakspace}
  \sphinxDUC{2500}{\sphinxunichar{2500}}
  \sphinxDUC{2502}{\sphinxunichar{2502}}
  \sphinxDUC{2514}{\sphinxunichar{2514}}
  \sphinxDUC{251C}{\sphinxunichar{251C}}
  \sphinxDUC{2572}{\textbackslash}
\fi
\usepackage{cmap}
\usepackage[T1]{fontenc}
\usepackage{amsmath,amssymb,amstext}
\usepackage{babel}



\usepackage{tgtermes}
\usepackage{tgheros}
\renewcommand{\ttdefault}{txtt}



\usepackage[Bjarne]{fncychap}
\usepackage{sphinx}

\fvset{fontsize=auto}
\usepackage{geometry}


% Include hyperref last.
\usepackage{hyperref}
% Fix anchor placement for figures with captions.
\usepackage{hypcap}% it must be loaded after hyperref.
% Set up styles of URL: it should be placed after hyperref.
\urlstyle{same}

\addto\captionsenglish{\renewcommand{\contentsname}{Contents:}}

\usepackage{sphinxmessages}
\setcounter{tocdepth}{1}



\title{Python P2P fileshare simulator}
\date{Apr 18, 2025}
\release{}
\author{Carter Ptak, Akira Takeuchi}
\newcommand{\sphinxlogo}{\vbox{}}
\renewcommand{\releasename}{}
\makeindex
\begin{document}

\ifdefined\shorthandoff
  \ifnum\catcode`\=\string=\active\shorthandoff{=}\fi
  \ifnum\catcode`\"=\active\shorthandoff{"}\fi
\fi

\pagestyle{empty}
\sphinxmaketitle
\pagestyle{plain}
\sphinxtableofcontents
\pagestyle{normal}
\phantomsection\label{\detokenize{index::doc}}


\sphinxAtStartPar
Add your content using \sphinxcode{\sphinxupquote{reStructuredText}} syntax. See the
\sphinxhref{https://www.sphinx-doc.org/en/master/usage/restructuredtext/index.html}{reStructuredText}
documentation for details.

\sphinxstepscope


\chapter{main module}
\label{\detokenize{main:module-main}}\label{\detokenize{main:main-module}}\label{\detokenize{main::doc}}\index{module@\spxentry{module}!main@\spxentry{main}}\index{main@\spxentry{main}!module@\spxentry{module}}\index{broadcast\_presence() (in module main)@\spxentry{broadcast\_presence()}\spxextra{in module main}}

\begin{fulllineitems}
\phantomsection\label{\detokenize{main:main.broadcast_presence}}
\pysigstartsignatures
\pysiglinewithargsret
{\sphinxcode{\sphinxupquote{main.}}\sphinxbfcode{\sphinxupquote{broadcast\_presence}}}
{}
{}
\pysigstopsignatures
\sphinxAtStartPar
Broadcast presence and local files on the network.

\end{fulllineitems}

\index{command\_line() (in module main)@\spxentry{command\_line()}\spxextra{in module main}}

\begin{fulllineitems}
\phantomsection\label{\detokenize{main:main.command_line}}
\pysigstartsignatures
\pysiglinewithargsret
{\sphinxcode{\sphinxupquote{main.}}\sphinxbfcode{\sphinxupquote{command\_line}}}
{}
{}
\pysigstopsignatures
\sphinxAtStartPar
Interactive command line for user input to manage the P2P file sharing.
This function provides a simple command line interface for users to list files,
view peers, request files, or exit the application.

\end{fulllineitems}

\index{download\_chunk() (in module main)@\spxentry{download\_chunk()}\spxextra{in module main}}

\begin{fulllineitems}
\phantomsection\label{\detokenize{main:main.download_chunk}}
\pysigstartsignatures
\pysiglinewithargsret
{\sphinxcode{\sphinxupquote{main.}}\sphinxbfcode{\sphinxupquote{download\_chunk}}}
{\sphinxparam{\DUrole{n}{filename}}\sphinxparamcomma \sphinxparam{\DUrole{n}{chunk\_num}}\sphinxparamcomma \sphinxparam{\DUrole{n}{owners}}\sphinxparamcomma \sphinxparam{\DUrole{n}{temp\_dir}}\sphinxparamcomma \sphinxparam{\DUrole{n}{download\_success}}}
{}
\pysigstopsignatures
\sphinxAtStartPar
Try to download a single chunk from any available owner.

\end{fulllineitems}

\index{download\_file() (in module main)@\spxentry{download\_file()}\spxextra{in module main}}

\begin{fulllineitems}
\phantomsection\label{\detokenize{main:main.download_file}}
\pysigstartsignatures
\pysiglinewithargsret
{\sphinxcode{\sphinxupquote{main.}}\sphinxbfcode{\sphinxupquote{download\_file}}}
{\sphinxparam{\DUrole{n}{filename}}}
{}
\pysigstopsignatures
\sphinxAtStartPar
Download a file chunk\sphinxhyphen{}by\sphinxhyphen{}chunk from multiple peers, assuming they have all chunks.

\end{fulllineitems}

\index{get\_my\_files() (in module main)@\spxentry{get\_my\_files()}\spxextra{in module main}}

\begin{fulllineitems}
\phantomsection\label{\detokenize{main:main.get_my_files}}
\pysigstartsignatures
\pysiglinewithargsret
{\sphinxcode{\sphinxupquote{main.}}\sphinxbfcode{\sphinxupquote{get\_my\_files}}}
{}
{}
\pysigstopsignatures
\sphinxAtStartPar
Get a list of available files and their total number of chunks and their full SHA\sphinxhyphen{}256 hash.

\end{fulllineitems}

\index{handle\_client() (in module main)@\spxentry{handle\_client()}\spxextra{in module main}}

\begin{fulllineitems}
\phantomsection\label{\detokenize{main:main.handle_client}}
\pysigstartsignatures
\pysiglinewithargsret
{\sphinxcode{\sphinxupquote{main.}}\sphinxbfcode{\sphinxupquote{handle\_client}}}
{\sphinxparam{\DUrole{n}{conn}}\sphinxparamcomma \sphinxparam{\DUrole{n}{addr}}}
{}
\pysigstopsignatures
\sphinxAtStartPar
Handle incoming file or chunk requests from peers.

\end{fulllineitems}

\index{list\_files() (in module main)@\spxentry{list\_files()}\spxextra{in module main}}

\begin{fulllineitems}
\phantomsection\label{\detokenize{main:main.list_files}}
\pysigstartsignatures
\pysiglinewithargsret
{\sphinxcode{\sphinxupquote{main.}}\sphinxbfcode{\sphinxupquote{list\_files}}}
{}
{}
\pysigstopsignatures
\sphinxAtStartPar
List all available files across all peers, based on their hash, excluding own files (by hash).

\end{fulllineitems}

\index{listen\_for\_peers() (in module main)@\spxentry{listen\_for\_peers()}\spxextra{in module main}}

\begin{fulllineitems}
\phantomsection\label{\detokenize{main:main.listen_for_peers}}
\pysigstartsignatures
\pysiglinewithargsret
{\sphinxcode{\sphinxupquote{main.}}\sphinxbfcode{\sphinxupquote{listen\_for\_peers}}}
{}
{}
\pysigstopsignatures
\sphinxAtStartPar
Listen for incoming peer discovery messages and update peer file lists.

\end{fulllineitems}

\index{serve\_files() (in module main)@\spxentry{serve\_files()}\spxextra{in module main}}

\begin{fulllineitems}
\phantomsection\label{\detokenize{main:main.serve_files}}
\pysigstartsignatures
\pysiglinewithargsret
{\sphinxcode{\sphinxupquote{main.}}\sphinxbfcode{\sphinxupquote{serve\_files}}}
{}
{}
\pysigstopsignatures
\sphinxAtStartPar
Serve files to peers that request them.
This function listens for incoming TCP connections on a specified port,
accepts file requests, and sends the requested files in chunks.
It runs in a separate thread to allow concurrent file serving.

\end{fulllineitems}

\index{sha256sum() (in module main)@\spxentry{sha256sum()}\spxextra{in module main}}

\begin{fulllineitems}
\phantomsection\label{\detokenize{main:main.sha256sum}}
\pysigstartsignatures
\pysiglinewithargsret
{\sphinxcode{\sphinxupquote{main.}}\sphinxbfcode{\sphinxupquote{sha256sum}}}
{\sphinxparam{\DUrole{n}{path}}}
{}
\pysigstopsignatures
\sphinxAtStartPar
Calculate the SHA\sphinxhyphen{}256 checksum of a file.
This function reads a file in chunks and computes its SHA\sphinxhyphen{}256 hash.
It is used to verify file integrity after download.

\end{fulllineitems}


\sphinxstepscope


\chapter{CSCI651\_fp}
\label{\detokenize{modules:csci651-fp}}\label{\detokenize{modules::doc}}
\sphinxstepscope


\section{conf module}
\label{\detokenize{conf:module-conf}}\label{\detokenize{conf:conf-module}}\label{\detokenize{conf::doc}}\index{module@\spxentry{module}!conf@\spxentry{conf}}\index{conf@\spxentry{conf}!module@\spxentry{module}}\begin{description}
\sphinxlineitem{command to generate documentation:}
\sphinxAtStartPar
sphinx\sphinxhyphen{}apidoc \sphinxhyphen{}f \sphinxhyphen{}o ./source ../
make make latexpdf

\end{description}


\renewcommand{\indexname}{Python Module Index}
\begin{sphinxtheindex}
\let\bigletter\sphinxstyleindexlettergroup
\bigletter{c}
\item\relax\sphinxstyleindexentry{conf}\sphinxstyleindexpageref{conf:\detokenize{module-conf}}
\indexspace
\bigletter{m}
\item\relax\sphinxstyleindexentry{main}\sphinxstyleindexpageref{main:\detokenize{module-main}}
\end{sphinxtheindex}

\renewcommand{\indexname}{Index}
\printindex
\end{document}